
\begin{description}

% \item[Geogebra] 

% \link{  .ggb}{GeoGebra applet}

\item[R] 

\link{http://www.math.unl.edu/~sdunbar1/ising234.R}{R script for
  complete Ising model.}

\begin{lstlisting}[language=R]
## WARNING: USE ONLY FOR N = 2, 3, 4

allBinarySeq <- function(N) { # USE ONLY FOR N <= 4!!
# list of N copies of -1, 1, simplify=FALSE returns lists, not vectors
# then all possible combinations of \pm 1 over N places
# resulting in data.frame of 2^N observations of N variables
# transpose to make N by 2^N matrix
# recast back to data.frame    
    as.data.frame(
        t(
            expand.grid(
                replicate(N, c(-1,1), simplify=FALSE)
            )
        )
    )
}

allConfigs <- function(N) { # USE ONLY FOR N <= 4
# returns LIST of all square configs
    X <- allBinarySeq(N*N)
    # data.frame of all 2^(N^2) observations over N^2 variables

    data = lapply(X, matrix, nrow=N, ncol=N)
}

downConfig <- function(N) { matrix(-1,N,N) } # 1 of 2 lowest energy configs
upConfig <- function(N) { matrix(1,N,N) }    # other lowest energy configs

energy  <-  function(config, J=1) {  #pass in a MATRIX
## J has units of energy
## Returns a potential energy, negative, with lowest energy -4N^2    
    N  <- NROW(config)

    below  <- config[ c(2:N, 1), ]
    left  <- config[ , c(N, 1:(N-1)) ]
    above  <- config[c(N, 1:(N-1)), ]
    right  <- config[ , c(2:N, 1) ]

    nearNeighProd <- config * (below + left + above + right)
    ## element-wise multiplication, note factorization of sum
    
    nrg  <- -J *sum(nearNeighProd)
}

allEnergies  <- function(configurations) { #list of configurations
    nrgs  <-  lapply(configurations, energy, J)
}

boltzWeight <- function(nrg, kTemp) { exp(- nrg/kTemp) }

netMag  <-  function(config) { abs( sum( config ) ) }

boltzDist  <- function(configurations, kTemperature) {
    bzs  <- unlist(
        lapply(allEnergies(configurations), boltzWeight, kTemperature)
    )
}

partitionFunc  <- function(configurations, kTemperature){ #list of energies
        Z  <- sum( boltzDist(configurations, kTemperature) )
}

netSpin <- function(configurations, N) {   #list of configurations
    lm <- lapply(configurations, matrix, N, N)
    sm  <- abs( unlist( lapply(lm, sum) ) ) #absolute value sum of entries in each matrix
    # unlist converts a list to a vector
}

magnetization  <- function(configurations, kTemperature) {
    Z <- partitionFunc(configurations, kTemperature)
    mag <- (1/Z) * sum(  netSpin(configurations, N) * boltzDist(configurations, kTemperature) )
}

J <- 1
N  <- 3

configs  <- allConfigs(N)

nKTemperatures  <-  11
Temperatures  <- seq(2,2.5, length=nKTemperatures)
# surrounding the critical value kT_c/J = 2.269 when J=1

m  <- rep(0, nKTemperatures)

for (i in 1:nKTemperatures) {
  m[i] = magnetization(configs, Temperatures[i])  
  cat("T = ", Temperatures[i], "mag = ",  m[i], "\n" )
}

plot(Temperatures, m)

\end{lstlisting}

\item[R] 

\link{http://www.math.unl.edu/~sdunbar1/isingMetropolis.R}{R script for
  Ising Metropolis model.}

\begin{lstlisting}
randomConfig <- function(N) {
  config <- matrix(2*sample(0:1, N*N, replace=TRUE)-1, N,N)
}

boltzWeight <- function(nrg, kBoltzT) { exp(- nrg/kBoltzT) }

surroundNeighbors  <- function(site) {
    belowSite  <- if (site[1] == N) c(1, site[2]) else c(site[1]+1, site[2])
    leftSite   <- if (site[2] == 1) c(site[1], N) else c(site[1], site[2]-1)
    aboveSite  <- if (site[1] == 1) c(N, site[2]) else c(site[1]-1, site[2])
    rightSite  <- if (site[2] == N) c(site[1], 1) else c(site[1], site[2]+1)
    neighbors  <- c(belowSite, leftSite, aboveSite, rightSite)
}

sweepMetropolis <- function(config) {
  for (x in 1:N) {
      for (y in 1:N) {
          site <- c(x,y)
          configAtSite  <- config[ site[1], site[2] ]
          configPrimeAtSite  <- configAtSite * (-1)

          neigh  <- surroundNeighbors( site ) 
          deltaEnergy <- -2*J * (configPrimeAtSite - configAtSite)*
              (config[ neigh[1], neigh[2] ] +
               config[ neigh[3], neigh[4] ] +
               config[ neigh[5], neigh[6] ]+
               config[ neigh[7], neigh[8] ]
              )
          if ( runif(1) <= min(1, boltzWeight(deltaEnergy, kBoltzT)) )
        {
        config[ site[1], site[2] ]  <- configPrimeAtSite
        }   

    ## IF deltaEnergy < 0, so energy(configPrime) < energy(config)
    ## so boltzWeight(deltaEnergy, kBoltzT) > 1,
    ## THEN  runif(1) is certain to be less than p so certainly
    ## change config to configPrime
    ## ELSE deltaEnergy > 0, so 
    ## energy(configPrime) > energy(config), so 
    ## boltzWeight(deltaEnergy, kBoltzT) < 1,
    ## so change on chance that runif(1) < p
    }
  }
  return(config)
}

J  <- 1
N  <- 100
kBoltzT  <- 2.2
mcmcSteps  <- 50

config  <- randomConfig(N)

for (i in 1:mcmcSteps ) {
    config  <- sweepMetropolis(config)
}

image(1:N, 1:N, config[N:1, ], 
      col=gray.colors(2),
      xlab="", ylab="", xaxt="n", yaxt="n")
## config[N:1, ] so that the plot corresponds to matrix order
## i.e. first row of matrix is at top of plot, last row at bottom
\end{lstlisting}

% \item[Octave]

% \link{http://www.math.unl.edu/~sdunbar1/    .m}{Octave script for .}

% \begin{lstlisting}[language=Octave]

% \end{lstlisting}

% \item[Perl] 

% \link{http://www.math.unl.edu/~sdunbar1/    .pl}{Perl PDL script for .}

% \begin{lstlisting}[language=Perl]

% \end{lstlisting}

% \item[SciPy] 

% \link{http://www.math.unl.edu/~sdunbar1/    .py}{Scientific Python script for .}

% \begin{lstlisting}[language=Python]

% \end{lstlisting}

\end{description}



%%% Local Variables:
%%% mode: latex
%%% TeX-master: t
%%% End:
