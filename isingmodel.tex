%%% -*-LaTeX-*-
%%% isingmodel.tex.orig
%%% Prettyprinted by texpretty lex version 0.02 [21-May-2001]
%%% on Tue Oct 18 08:55:02 2022
%%% for Steve Dunbar (sdunbar@family-desktop)

\documentclass[12pt]{article}

\input{../../../../etc/macros} %\input{../../../../etc/mzlatex_macros}
\input{../../../../etc/pdf_macros}

\bibliographystyle{plain}

\begin{document}

\myheader \mytitle

\hr

\sectiontitle{The Ising Model}

\hr

\usefirefox

\hr

% \visual{Study Tip}{../../../../CommonInformation/Lessons/studytip.png}
% \section*{Study Tip}

% \hr

\visual{Rating}{../../../../CommonInformation/Lessons/rating.png}
\section*{Rating} %one of
% Everyone: contains no mathematics.
% Student: contains scenes of mild algebra or calculus that may require guidance.
Mathematically Mature:  may contain mathematics beyond calculus with
proofs.  % Mathematicians Only: prolonged scenes of intense rigor.

\hr

\visual{Section Starter Question}{../../../../CommonInformation/Lessons/question_mark.png}
\section*{Section Starter Question}

What is a phase transition?  What causes phase transitions?  What are
some examples of phase transitions?

\hr

\visual{Key Concepts}{../../../../CommonInformation/Lessons/keyconcepts.png}
\section*{Key Concepts}

\begin{enumerate}
    \item
        The Ising model is a lattice model of spins for atoms with
        magnetic dipoles either up or down in some fixed direction.  The
        Ising model exhibits a \defn{phase transition} in the
        magnetization, a dramatic change in state as a parameter passes
        through a critical value.
    \item
        It is impractical to directly compute the expected magnetization
        since the number of configurations is too large.
    \item
        Estimates on rates of convergence indicate slow convergence for
        the Ising model Metropolis algorithm considered here.  The slow
        convergence is consistent with the number of steps needed to
        ensure visiting most sites.
\end{enumerate}

\hr

\visual{Vocabulary}{../../../../CommonInformation/Lessons/vocabulary.png}
\section*{Vocabulary}
\begin{enumerate}
    \item
        The \defn{Ising model} is a simple model of a ferromagnet.
    \item
        A \defn{phase transition} is a dramatic change in state as a
        parameter passes through a critical value.
    \item
        The \defn{Boltzmann weight} \(
        \operatorname{Boltz}
        (\omega) = \EulerE^{-E(\omega)/(\text{k}_{\text{Boltz}} T)} \)
        where \( \text{k}_{\text{Boltz}} \) is the Boltzmann constant,
        so \( \text{k}_{\text{Boltz}} T \) has units of energy.
    \item
        For a configuration \( \omega \) let
        \[
            M(\omega) = \sum\limits_{i=1}^N \omega_{i}.
        \]
    \item
        The \defn{magnetization} at temperature \( T \) is \( \E{M}_T \)
        is the expected value of \( M(\omega) \).
    \item
        \defn{Glauber dynamics} is the algorithm that selects sites at
        random in the lattice and computes magnetization in a Markov
        chain.
\end{enumerate}

\hr

\section*{Notation}
\begin{enumerate}
    \item
        \( T_c \) -- the temperature above which certain materials lose
        their permanent magnetic properties
    \item
        \( \text{k}_{\text{Boltz}} \) -- the Boltzmann constant from
        statistical mechanics
    \item
        \( J \) -- a proportionality constant for the energy of
        alignment
    \item
        \( N \) -- number of rows and columns in \( 2 \)-dimensional
        lattice with \( N^2 \) sites
    \item
        \( (i,j) \) -- lattice site
    \item
        \( \omega_{i,j} = \pm 1 \) -- spin at site \( (i,j) \)
    \item
        \( \omega = (\omega_{1,1}, \omega_{1,2}, \dots, \omega_{N,N}) \)
        -- configuration as a matrix of spins
    \item
        \( \Omega \) -- the set of all possible configurations
    \item
        \( E(\omega) \) -- magnetic energy of a configuration
    \item
        \( H \ge 0 \) -- the external magnetic field
    \item
        \( \langle i, j \rangle \) -- the set of \( 4 \) sites that are
        periodic nearest neighbors
    \item
        \( \text{k}_{\text{Boltz}} \) -- Boltzmann constant
    \item
        \(
        \operatorname{Boltz}
        (\omega) = \EulerE^{-E(\omega)/(\text{k}_{\text{Boltz}} T)} \)
        -- Boltzmann weight
    \item
        \( M(\omega) \) -- net spin
    \item
        \( \E{M}_T \) -- the expected value of \( M(\omega) \)
    \item
        \( z = \sum\limits_{\omega \in \Omega} \EulerE{-E(\omega)/(\text
        {k}_{\text{Boltz}} T)} \) -- the partition function
    \item
        \( (i,j) \) -- random site in the matrix
    \item
        \( \omega'_{(i,j)} \) -- flipped sign value at \( (i,j) \)
    \item
        \( \Delta E \) -- change in energy from flip

\end{enumerate}

\visual{Mathematical Ideas}{../../../../CommonInformation/Lessons/mathematicalideas.png}
\section*{Mathematical Ideas}

Magnets are common objects but magnetism itself is enigmatic.  Heating a
ferromagnetic material such as iron causes it to lose magnetism.  Slow
cooling allows the material to magnetize again.  From 1922 to 1925
Wilhelm Lenz and his student Ernst Ising developed what is now called
the Ising model to explain this phase transition in magnetism caused by
heating or cooling.  The Ising model connects electromagnetic theory
with statistical mechanics.  Exploring the Ising model for the magnetic
phase transition uses Markov chains.  The goal of this section is to
describe applications of the Metropolis algorithm to the Ising model for
magnetism.

\subsection*{Magnetism and the Ising Model}

In a magnetic material such as iron, the atoms each have a magnetic
moment.  A piece of iron has small regions or ``domains'' that are
completely magnetized because the atoms align in the same direction. The
overall energy of neighboring atoms in the domain is lower when the
moments are parallel.  If the energy of neighboring atoms and domains
were the only consideration, a magnetic material would settle into a
magnetized configuration, but heat tends to randomize the magnetic
moment directions.  On a large scale, the directions of the domains are
not aligned so average magnetization is about zero.

In the physics of magnetism, the Curie temperature \( T_c \)%
\index{Curie temperature}
is the temperature above which certain materials lose their permanent
magnetic properties.  The Curie temperature is named after Pierre Curie,
who showed that magnetism disappears above a critical temperature. Above
\( T_c \) thermal energy disrupts the tendency toward magnetic dipole
alignment, and materials become just weakly magnetic.

The \defn{Ising model}%
\index{Ising model}
is a simple model of a ferromagnet's domain alignment due to energy and
heat.  The model describes the tendency for neighboring atoms to align
their magnetic fields with each other or with an external magnetic
field.  The Ising model exhibits a \defn{phase transition}%
\index{phase transition}
in the magnetization, a dramatic change between differently ordered
magnetic states as a parameter passes through a critical value.
Moreover, the \( 2 \)-dimensional Ising model is one of the most
important theoretical statistical mechanical models.  As with all
mathematical models, the aim is to find the simplest framework that
exhibits important properties of the physical system.  With just the
simple elements of binary variables on a regular grid, the Ising model
exhibits phase transitions.

The Ising model is a lattice model of atomic magnetic dipoles aligned up
or down in some fixed direction.  A \( 1 \)-dimensional Ising model
pictures dipoles regularly spaced along a line.  A \( 2 \)-dimensional
Ising model is a square lattice, and a \( 3 \)-dimensional Ising model
is a cubic lattice.  Each atom interacts with its nearest neighbors. The
goal of the model is to find \( T_c \).  Interestingly, the \( 1 \)-dimensional
Ising model does not have a phase transition, first proved by Ernst
Ising in 1925.  A 1944 result of Lars Onsager gives a relation for the
exact value for the critical temperature in the \( 2 \)-dimensional
Ising model:
\[
    \frac{\text{k}_{\text{Boltz}} T_c}{J} = \frac{2}{\log(1 + \sqrt{2})}
    \approx 2.269
\] where \( \text{k}_{\text{Boltz}} \) is the Boltzmann constant from
statistical mechanics and \( J \) is a proportionality constant for the
energy of a dipole alignment.

For definiteness for this application of the Metropolis algorithm,
assume a planar square lattice with \( N^2 \) sites.  An \( N \times N \)
matrix represents the square lattice, the first row at the top, and the \(
N \)th row at the bottom, the first column at the left, the \( N \)th
column at the right.  At each lattice site \( (i,j) \), there is a \emph
{spin} represented by \( \omega_{i,j} = \pm 1 \).  A configuration is a
matrix of spins, \( \omega = (\omega_ {1,1}, \omega_{1,2}, \dots, \omega_
{N,N}) \).%
\index{configuration}
Define \( \Omega \) as the set of all possible configurations so \(
\card{\Omega} = 2^{N^2} \).  The magnetic energy (also called the
Hamiltonian) of a configuration is
\[
    E(\omega) = -J \sum\limits_{i, j=1}^{N} \sum\limits_{k=1}^4 \omega_{i,
    j} \omega_{\left\langle i,j \right\rangle[k]} - H \sum\limits_{i=1}^N
    \omega_i
\] where \( J > 0 \) is the nearest-neighbor affinity (a proportionality
constant with units of energy for the strength of the binding between
neighbors), \( H \ge 0 \) represents the external magnetic field and \(
\langle i, j \rangle \) indicates the set of \( 4 \) sites that are
periodic nearest neighbors, sites in the lattice with a horizontal or
vertical bond.  For the top row with \( i = 1 \), periodic means \(
\left\langle 1, j \right\rangle = \set{(2,j), (1, j-1), (N,j), (1, j+1)}
\), interpreting periodically if \( j=1 \) or \( j=N \).  Similarly for
the bottom row, and the leftmost and rightmost columns.  The periodic
boundary conditions mean that boundary sites are not treated differently
than interior sites and make the finite grid simulate an infinite grid.
For purposes of the example, assume there is no external field, so \(
H=0 \) and that \( J=1 \).

The probability distribution of spin configurations is proportional to
the \defn{Boltzmann weight}%
\index{Boltzmann weight}
\(
\operatorname{Boltz}
(\omega) = \EulerE^{-E(\omega)/(\text{k}_{\text{Boltz}} T)} \) where \(
\text{k}_{\text{Boltz}} \) is the Boltzmann constant and \( T \) is the
absolute temperature, so \( \text{k}_{\text{Boltz}} T \) has units of
energy.  The energy units are chosen to make \( \text{k}_{\text{Boltz}}
T_C = \frac{2}{\log(1 + \sqrt{2})} \).  This section uses the
combination \( \text{k}_{\text{Boltz}} T \) consistently for the heat
energy at temperature \( T \) to emphasize the connection with
statistical mechanics, with \( 0 \le \text{k}_{\text{Boltz}} T \le 10 \)
for illustration. Given a configuration, the Boltzman weight is
proportional to the probability that configuration will appear at
temperature \( T \).  For completely ordered configurations with \(
\omega_{i,j} \equiv +1 \) or \( \omega_{i,j} \equiv -1 \), \( E(\omega)
= -J \cdot 4N^2 \) (low magnetization energy) and \(
\operatorname{Boltz}
(\omega) = \EulerE^{J \cdot 4N^2/(\text{k}_{\text{Boltz}} T)} \) is
large (relatively higher probability).  For highly disordered
configurations with \( \omega_{i,j} = \pm 1 \) distributed randomly, \(
E(\omega) \approx 0 \) (higher magnetization energy) and \(
\operatorname{Boltz}
(\omega) \approx \EulerE^{-0/(\text{k}_{\text{Boltz}} T)} \approx 1 \)
is small (relatively lower probability).

For a configuration \( \omega \) the net spin is
\[
    M(\omega) = \sum\limits_{i=1}^N \omega_{i}.
\] The \defn{magnetization}%
\index{magnetization}
at temperature \( T \) is \( \E{M}_T \) is the expected value of \( M(\omega)
\) with respect to the Boltzmann weight:
\begin{align*}
    \E{M}_T &= \sum\limits_{\omega \in \Omega} M(\omega)
    \operatorname{Boltz}
    (\omega) \\
    &= \frac{1}{z} \sum\limits_{\omega \in \Omega} M(\omega) \EulerE^{-E
    (\omega)/(\text{k}_{\text{Boltz}} T)}.
\end{align*}
Recall that the denominator
\[
  z = \sum\limits_{\omega \in \Omega}
           \EulerE^{-E(\omega)/(\text{k}_{\text{Boltz}} T)}
\] is called the partition function.%
\index{partition function}

This model exhibits some general features.  At high temperatures there
is no correlation between sites, states are essentially uniformly
distributed and hence \( \E{M}_T \) is near zero.  This highly
disordered configuration is not magnetic.  Lowering the temperature to a
critical value called the Curie temperature \( T_c \), spontaneous
magnetization occurs.  The model goes to one of the highly ordered state
of all \( +1 \) or all \( -1 \) spins.  See Figure~%
\ref{fig:isingmodel:phasediagram} for a schematic diagram of the phase
transition.  The Curie temperature is a critical temperature below which
sites influence each other at long distances.  It is impractical to
directly compute the expected magnetization since the number of
configurations is too large.  Even for \( N=5 \), to directly compute
the magnetization requires \( 2^{25} \) matrices with \( 25 \) entries,
over \( 838 \) million values.

\begin{figure}
    \centering
\begin{asy}
import graph;

size(5inches);

real myfontsize = 12;
real mylineskip = 1.2*myfontsize;
pen mypen = fontsize(myfontsize, mylineskip);
defaultpen(mypen);

real Tc = 2;

real mplus(real x) {return (Tc - x)^(1/8);}
real mminus(real x) {return -mplus(x);}

draw(graph(mplus, 0, Tc), red+linewidth(2));
draw(graph(mminus, 0, Tc), red+linewidth(2));
draw((Tc,0)--(Tc+2,0), red+linewidth(2));
draw((Tc,-1)--(Tc,+1), dotted);

xaxis("Temperature", xmin=0, xmax=4.5, Arrow); 
yaxis("Magnetization", ymin=-1.5, ymax=1.5, Arrows); 

label("All $+1$", (0,mplus(0)), NE);
label("All $-1$", (0,mminus(0)), SE);
label("$T_c$", (Tc,0), SE);
\end{asy}
    \caption{Schematic magnetic phase diagram for the Ising model.}%
    \label{fig:isingmodel:phasediagram}
\end{figure}

\subsection*{The Metropolis Algorithm and the Ising Model}

Since direct calculation is unfeasible, the goal is to apply the
Metropolis algorithm to the Ising model to find \( T_c \).  To do that,
run the Markov chain Monte Carlo method at different temperatures to
find the average magnetization as a function of temperature to find
where the phase change occurs. Figure~%
\ref{fig:isingmodel:isingbelowtc} and Figure~%
\ref{fig:isingmodel:isingabovetc} are snapshots of applying the
Metropolis algorithm to the Ising model. Figure~%
\ref{fig:isingmodel:isingbelowtc} shows large areas of consistent
magnetization after \( 500{,}000 \) Markov chain steps started from a
random configuration at a normalized temperature \( 2.2 < T_c \).
Figure~%
\ref{fig:isingmodel:isingbelowtc} appears random even after \( 50{,}000{,}000
\) Markov chain steps started from a random configuration at a
temperature \( 5.0 > T_c \).

\begin{figure}
    \centering
        \includegraphics[scale=0.50]{isingR_N100_T22_Step50}
        \caption{A \( 100 \times 100 \) Ising model at \( T = 2.2 \)
        after \( 500{,}000 \) steps.}%
        \label{fig:isingmodel:isingbelowtc}
\end{figure}
\begin{figure}      
        \includegraphics[scale=0.50]{isingR_N100_T5_Step5000}
        \caption{A \( 100 \times 100 \) Ising model at \( T = 5.0 \)
        after \( 50{,}000{,}000 \) steps.}%
        \label{fig:isingmodel:isingabovetc}
\end{figure}

The key idea behind the algorithm is to emphasize the low energy subset
of configurations and discount the rest.  The Metropolis algorithm%
\index{Metropolis algorithm}
takes a random walk through the configuration space, visiting more
frequently occurring spin configurations more often.  In the words of
the original Metropolis algorithm paper:  ``Instead of choosing
configurations randomly, then weighting them with \( \EulerE^{-E/(\text{k}_
{\text{Boltz}} T)} \), instead choose configurations with probability \(
\EulerE^{-E/(\text {k}_{\text{Boltz}} T)} \) and weight them evenly''.
The Metropolis algorithm for the Ising model proceeds as follows:
\begin{enumerate}
    \item
        Visit all lattice sites \( (i, j) \) in some deterministic
        order. For example, visit across the first row, then across the
        second row, and so on to the last row.
    \item
        As each site is visited, calculate the change in energy that
        would result from flipping the spin at the selected site.  If \(
        \omega \) is the visited configuration and \( \omega' \) is the
        configuration with the flipped spin at the visited site, then
        \[
            \Delta E = E(\omega') - E(\omega) = -2J (\omega'_{(i,j)} -
            \omega_{(i,j)}) \sum\limits_{k=1}^4\omega_{\left\langle i,j
            \right\rangle[k]}
        \] where the sum is only over the four nearest neighbors of the \(
        (i, j) \) site.  The computational cost of updating a site is
        both small and independent of the size of the lattice.
    \item
        The acceptance rule is: if energy increases, \( E(\omega') > E(\omega)
        \) or equivalently \(
        \operatorname{Boltz}
        (\omega') <
        \operatorname{Boltz}
        (\omega') \), let \( \omega^{\star} = \omega' \) with acceptance
        probability
        \[
            \frac{%
            \operatorname{Boltz}
            (\omega')}{%
            \operatorname{Boltz}
            (\omega)} = \EulerE^{-\Delta E/\text{k}_{\text{Boltz}} T}.
        \]
    \item
        Otherwise, energy decreases, so accept the change, \( \omega^{\star}
        = \omega' \).
\end{enumerate}
Verifying the Markov chain of configuration changes is irreducible and
aperiodic is routine.

The alternative Glauber dynamics algorithm%
\index{Glauber dynamics algorithm}
(or in a different context the \emph{random chain algorithm}%
\index{random chain algorithm}%
) is:
\begin{enumerate}
    \item
        The visitation rule between configurations is to pick a lattice
        site \( (i, j) \) uniformly at random from \( i, j = 1, 2, 3,
        \dots, N \).
    \item
        As each site is visited, calculate the change in energy that
        would result from flipping the spin at the selected site.  If \(
        \omega \) is the visited configuration and \( \omega' \) is the
        configuration with the flipped spin at the visited site, then
        \[
            \Delta E = E(\omega') - E(\omega) = -2J (\omega'_{(i,j)} -
            \omega_{(i,j)}) \sum\limits_{k=1}^4\omega_{\left\langle i,j
            \right\rangle[k]}
        \] where the sum is only over the four nearest neighbors of the \(
        (i, j) \) site.  The computational cost of updating a site is
        both small and independent of the size of the lattice.
    \item
        The current configuration \( \omega \) becomes the flipped site
        configuration \( \omega' \) with acceptance probability
        \[
            \frac{\EulerE^{-\Delta E/\text{k}_{\text{Boltz}} T}}{1 +
            \EulerE^{-\Delta E/\text{k}_{\text{Boltz}} T}}.
        \]
    \item
        Otherwise, \( \omega^{\star} = \omega \).
\end{enumerate}
Verifying the Markov chain of configuration changes is irreducible and
aperiodic is routine.

Each algorithm has advantages and disadvantages.  For the Metropolis
algorithm at low temperatures unfavorable flips are rare, resulting in a
(nearly) deterministic flip rule along a deterministic path.  This can
result in the algorithm getting trapped in patterns in the lattice.  On
the other hand, with random sampling of sites in the Glauber algorithm,
taking \( 10{,}000 \) steps on a \( 100 \times 100 \) lattice will miss
about \( 3 {,}679 = 10{,}000/\EulerE \) sites, see the exercises.
Therefore, it takes more steps than the size of the lattice to ensure
visiting all sites.  Also, the acceptance probability in the Glauber
algorithm is less than the acceptance probability in the Metropolis
algorithm, see the exercises.  So the Glauber algorithm converges more
slowly than the Metropolis algorithm. The survey article by Diaconis and
Saloff
\cite{DIACONIS199820} presents estimates on rates of convergence
indicating slow convergence for the Ising model algorithms considered
here.

The two scripts below illustrate the global approach to the Ising model
and the Monte Carlo Markov chain method.  The script \texttt{isingDist.R}
calculates the entire configuration space, the individual energy of each
configuration and the partition function.  However it is only practical
for \( N = 2, 3, 4 \), not refined enough to demonstrate the phase
transition.  The second script \texttt{isingMetropolis.R} illustrates the
Monte Carlo Markov chain approach.  The script directly implements the
Metropolis method above but is not general.  For a general
approach, see the R package \texttt{IsingSampler}
\cite{epskamp20} and for another implementation in Python, see
\cite{schlusser18}.

The comprehensive and very readable survey article
\cite{hayes21} provides interactive graphics for comparing the
Metropolis and Glauber algorithms directly, mixing the visitation and
acceptance rules for the Metropolis and Glauber algorithms, using
different visitation rules for the Metropolis algorithm, and testing the
effects of different initial and boundary conditions.  The article also
has figures for the Boltzman weight function at various temperatures,
comparison of the acceptance functions, comparisons of rates of
convergence, and explanations for the growth of shapes in the graphics.
Finally, the article has a well-researched history of the Metropolis and
Glauber algorithms.

%% https://rajeshrinet.github.io/blog/2014/ising-model/
%% https://github.com/RaewynD/IsingModel/blob/master/lattice.py

%% https://criticathink.wordpress.com/2018/07/15/ising-model-simulation-in-r-using-the-metropolis-monte-carlo-algorithm/
%% https://cran.r-project.org/web/packages/IsingSampler/index.html
%% https://tanyaschlusser.github.io/posts/mcmc-and-the-ising-model/

\visual{Section Starter Question}{../../../../CommonInformation/Lessons/question_mark.png}
\section*{Section Ending Answer}

A phase transition is dramatic change in state as a parameter passes
through a critical value.  In physical and chemical systems, the
transition occurs from the addition or removal of energy, usually in the
form of heat, but it can also result from pressure changes or other
causes.  Examples of phase transitions are the freezing of water into
ice, or the reverse melting, the vaporization of water to steam or the
reverse condensation.  Biological system exhibit phase transitions at
the molecular, cellular, and even the organism level.  In card-shuffling
the rapid transition from relatively ordered to completely disordered
after \( O(n \log n) \) shuffles is an example of a mathematical phase
transition.

\subsection*{Sources} Ideas about the Ising model and the associated
Metropolis algorithm are adapted from
\cite{hayes19, hayes21, richey10, schlusser18}. The exercises are
inspired by
\cite{hayes21}.

\nocite{}
\nocite{}

\hr

\visual{Algorithms, Scripts, Simulations}{../../../../CommonInformation/Lessons/computer.png}
\section*{Algorithms, Scripts, Simulations}

\subsection*{Algorithm}

\begin{algorithm}
  \DontPrintSemicolon
  \KwData{Dimension \( N \)}
  \KwResult{Plot of magnetization versus normalized temperature near \(
    T_c \)}
  
  \SetKwProg{Fn}{function}{}{}
  \SetKwFunction{allBinarySeq}{allBinarySeq}
  \SetKwFunction{allConfigs}{allConfigs}
  \SetKwFunction{energy}{energy}
  \SetKwFunction{allEnergies}{allEnergies}
  \SetKwFunction{boltzWeight}{boltzWeight}
  \SetKwFunction{netMag}{netMag}
  \SetKwFunction{boltzDist}{boltzDist}
  \SetKwFunction{partitionFunct}{partitionFunct}
  
  \Fn{allBinarySeq(N)}{creates all \( 2^N \) combinations of \( \pm
  1 \) over \( N \) places\;}
  \Fn{allConfigs(N)}{creates a list of all possible square
  configurations of \( \pm 1 \) of size \( N \times N \) using
  \allBinarySeq{N*N}\;}
  \Fn{energy(config)}{computes magnetization energy\;}
  \Fn{allEnergies}{applies energy function across list of all
  energies\;}
  \Fn{boltzWeight(energy, kTemp)}{computes \(\EulerE^{energy*
    kTemp} \)\;}
  \Fn{netMag(config)}{computes net magnetization\;}
  \Fn{boltzDist}{computes list of all \boltzWeight{energy, kTemp}\;}
  \Fn{partitionFunct(configurations, kTemp )}{computes sum of all
  energies using \boltzDist{configurations, kTemp}\;}
  \BlankLine\;

  Set \( J \leftarrow 1 \)\;
  Set \( N \leftarrow 3 \)\;
  With \allConfigs{N}, make a list of all configurations of size \(
  N \times N \)\;
  \BlankLine\;
  Set number of normalized temperatures to compute\;
  Set a list of normalized temperatures of that
  length from \( (2, 2.5) \) covering the critical value of \( T_c = 2.269 \) when
  \( J = 1 \)\;
  Compute the magnetization over all configurations for each of the
  normalized temperatures\;
  Plot the magnetization as a function of normalized temperature\;
\caption{Ising model for magnetization for sizes \( 2, 3, 4\).}  
\end{algorithm}

\begin{algorithm}
  \DontPrintSemicolon
  \KwData{Dimension \( N \), normalized temperature, number of sweeps}
  \KwResult{Plot of magnetization after many steps of the Metropolis algorithm}

  \SetKwProg{Fn}{function}{}{}
  \SetKwFunction{randomConfig}{randomConfig}
  \SetKwFunction{boltzWeight}{boltzWeight}
  \SetKwFunction{surroundNeighbors}{surroundNeighbors}
  \SetKwFunction{sweepMetropolis}{sweepMetropolis}
  \SetKwFunction{Unif}{Unif}

  \Fn{Unif(n)}{creates a sample of \(n\) uniform random variables from \((0,1)
  \)}
  \Fn{randomConfig(N)}{creates a random \( N \times N \) configuration of
    \( \pm 1 \)}
  \Fn{boltzWeight(energy, kTemp)}{computes \(\EulerE^{-energy/
      kTemp} \)\;}
  \Fn{surroundNeighbors(site)}{find the \( 4 \) periodic surrounding
    neighbors of \( site \)}
  \Fn{sweepMetropolis(config, N)}{
    \For{\(x \leftarrow 1:N \)} {
      \For{ \( y \leftarrow 1:N \)} {
        \(configAtSite  \leftarrow config[ x, y ] \)\;
          \(configPrimeAtSite  \leftarrow configAtSite * (-1) \)\;
          \( neigh  \leftarrow \surroundNeighbors{site} \)\; 
          Compute \( deltaEnergy \) using \( neigh \) \; 
          \uIf {\( Unif(1) < \min(1, boltzWeight(deltaEnergy, kTemperature)) \)}{
              \( config \leftarrow configPrime \)\;}
          }
      }
   }
    
  
  \BlankLine\;
  Set \( J \leftarrow 1 \),  \( kTemp \),  \( N \),  \(mcmcSteps\)\;
  Choose random configuration with \randomConfig{N}\;
  \For{\( 1:mcmcSteps \)}{
    \( config \leftarrow \) \sweepMetropolis{config, N}\;
  }
  Plot image of resulting configuration
  \caption{Metropolis algorithm applied to Ising model.}
\end{algorithm}

\subsection*{Scripts}


\begin{description}

% \item[Geogebra] 

% \link{  .ggb}{GeoGebra applet}

\item[R] 

\link{http://www.math.unl.edu/~sdunbar1/ising234.R}{R script for
  complete Ising model.}

\begin{lstlisting}[language=R]
## WARNING: USE ONLY FOR N = 2, 3, 4

allBinarySeq <- function(N) { # USE ONLY FOR N <= 4!!
# list of N copies of -1, 1, simplify=FALSE returns lists, not vectors
# then all possible combinations of \pm 1 over N places
# resulting in data.frame of 2^N observations of N variables
# transpose to make N by 2^N matrix
# recast back to data.frame    
    as.data.frame(
        t(
            expand.grid(
                replicate(N, c(-1,1), simplify=FALSE)
            )
        )
    )
}

allConfigs <- function(N) { # USE ONLY FOR N <= 4
# returns LIST of all square configs
    X <- allBinarySeq(N*N)
    # data.frame of all 2^(N^2) observations over N^2 variables

    data = lapply(X, matrix, nrow=N, ncol=N)
}

downConfig <- function(N) { matrix(-1,N,N) } # 1 of 2 lowest energy configs
upConfig <- function(N) { matrix(1,N,N) }    # other lowest energy configs

energy  <-  function(config, J=1) {  #pass in a MATRIX
## J has units of energy
## Returns a potential energy, negative, with lowest energy -4N^2    
    N  <- NROW(config)

    below  <- config[ c(2:N, 1), ]
    left  <- config[ , c(N, 1:(N-1)) ]
    above  <- config[c(N, 1:(N-1)), ]
    right  <- config[ , c(2:N, 1) ]

    nearNeighProd <- config * (below + left + above + right)
    ## element-wise multiplication, note factorization of sum
    
    nrg  <- -J *sum(nearNeighProd)
}

allEnergies  <- function(configurations) { #list of configurations
    nrgs  <-  lapply(configurations, energy, J)
}

boltzWeight <- function(nrg, kTemp) { exp(- nrg/kTemp) }

netMag  <-  function(config) { abs( sum( config ) ) }

boltzDist  <- function(configurations, kTemperature) {
    bzs  <- unlist(
        lapply(allEnergies(configurations), boltzWeight, kTemperature)
    )
}

partitionFunc  <- function(configurations, kTemperature){ #list of energies
        Z  <- sum( boltzDist(configurations, kTemperature) )
}

netSpin <- function(configurations, N) {   #list of configurations
    lm <- lapply(configurations, matrix, N, N)
    sm  <- abs( unlist( lapply(lm, sum) ) ) #absolute value sum of entries in each matrix
    # unlist converts a list to a vector
}

magnetization  <- function(configurations, kTemperature) {
    Z <- partitionFunc(configurations, kTemperature)
    mag <- (1/Z) * sum(  netSpin(configurations, N) * boltzDist(configurations, kTemperature) )
}

J <- 1
N  <- 3

configs  <- allConfigs(N)

nKTemperatures  <-  11
Temperatures  <- seq(2,2.5, length=nKTemperatures)
# surrounding the critical value kT_c/J = 2.269 when J=1

m  <- rep(0, nKTemperatures)

for (i in 1:nKTemperatures) {
  m[i] = magnetization(configs, Temperatures[i])  
  cat("T = ", Temperatures[i], "mag = ",  m[i], "\n" )
}

plot(Temperatures, m)

\end{lstlisting}

\item[R] 

\link{http://www.math.unl.edu/~sdunbar1/isingMetropolis.R}{R script for
  Ising Metropolis model.}

\begin{lstlisting}
randomConfig <- function(N) {
  config <- matrix(2*sample(0:1, N*N, replace=TRUE)-1, N,N)
}

boltzWeight <- function(nrg, kBoltzT) { exp(- nrg/kBoltzT) }

surroundNeighbors  <- function(site) {
    belowSite  <- if (site[1] == N) c(1, site[2]) else c(site[1]+1, site[2])
    leftSite   <- if (site[2] == 1) c(site[1], N) else c(site[1], site[2]-1)
    aboveSite  <- if (site[1] == 1) c(N, site[2]) else c(site[1]-1, site[2])
    rightSite  <- if (site[2] == N) c(site[1], 1) else c(site[1], site[2]+1)
    neighbors  <- c(belowSite, leftSite, aboveSite, rightSite)
}

sweepMetropolis <- function(config) {
  for (x in 1:N) {
      for (y in 1:N) {
          site <- c(x,y)
          configAtSite  <- config[ site[1], site[2] ]
          configPrimeAtSite  <- configAtSite * (-1)

          neigh  <- surroundNeighbors( site ) 
          deltaEnergy <- -2*J * (configPrimeAtSite - configAtSite)*
              (config[ neigh[1], neigh[2] ] +
               config[ neigh[3], neigh[4] ] +
               config[ neigh[5], neigh[6] ]+
               config[ neigh[7], neigh[8] ]
              )
          if ( runif(1) <= min(1, boltzWeight(deltaEnergy, kBoltzT)) )
        {
        config[ site[1], site[2] ]  <- configPrimeAtSite
        }   

    ## IF deltaEnergy < 0, so energy(configPrime) < energy(config)
    ## so boltzWeight(deltaEnergy, kBoltzT) > 1,
    ## THEN  runif(1) is certain to be less than p so certainly
    ## change config to configPrime
    ## ELSE deltaEnergy > 0, so 
    ## energy(configPrime) > energy(config), so 
    ## boltzWeight(deltaEnergy, kBoltzT) < 1,
    ## so change on chance that runif(1) < p
    }
  }
  return(config)
}

J  <- 1
N  <- 100
kBoltzT  <- 2.2
mcmcSteps  <- 50

config  <- randomConfig(N)

for (i in 1:mcmcSteps ) {
    config  <- sweepMetropolis(config)
}

image(1:N, 1:N, config[N:1, ], 
      col=gray.colors(2),
      xlab="", ylab="", xaxt="n", yaxt="n")
## config[N:1, ] so that the plot corresponds to matrix order
## i.e. first row of matrix is at top of plot, last row at bottom
\end{lstlisting}

% \item[Octave]

% \link{http://www.math.unl.edu/~sdunbar1/    .m}{Octave script for .}

% \begin{lstlisting}[language=Octave]

% \end{lstlisting}

% \item[Perl] 

% \link{http://www.math.unl.edu/~sdunbar1/    .pl}{Perl PDL script for .}

% \begin{lstlisting}[language=Perl]

% \end{lstlisting}

% \item[SciPy] 

% \link{http://www.math.unl.edu/~sdunbar1/    .py}{Scientific Python script for .}

% \begin{lstlisting}[language=Python]

% \end{lstlisting}

\end{description}



%%% Local Variables:
%%% mode: latex
%%% TeX-master: t
%%% End:


\hr

\visual{Problems to Work}{../../../../CommonInformation/Lessons/solveproblems.png}
\section*{Problems to Work for Understanding}

\renewcommand{\theexerciseseries}{}
\renewcommand{\theexercise}{\arabic{exercise}}

\begin{exercise}
    Show that with random sampling of sites, taking \( 10{,}000 \) steps
    on a \( 100 \times 100 \) lattice will miss about \( 3{,}679 = 10{,}000/\EulerE
    \) sites.
\end{exercise}
\begin{solution}
    With \( n \) sites, the chance of missing a site is \( 1 - \frac{1}{n}
    \).  In \( n \) iterations of the selection, the chance of missing
    the site is \( (1- \frac{1}{n})^n \approx 1/\EulerE \) for \( n \)
    large.  Among \( n \) sites there will be approximately \( 1/\EulerE
    \cdot n \) sites missed.
\end{solution}

\begin{exercise}
    With the nearest neighbor energy model, show that \( \Delta E \) can
    only have the values \( -16, -8, 0, +8, +16 \).  (Use the convention
    that \( J = 1 \).)
\end{exercise}
\begin{solution}
    \[
        \Delta E = E(\omega') - E(\omega) = -2J (\omega'_{(i,j)} -
        \omega_{(i,j)}) \sum\limits_{k=1}^4\omega_{\left\langle i,j
        \right\rangle[k]}
    \] where the sum is only over the four nearest neighbors of the \( (i,
    j) \) site.  The difference \( (\omega'_{(i,j)} - \omega_{(i,j)}) =
    \pm 2 \).  The sum over the four nearest neighbors will be \( \pm 4 \)
    if all \( 4 \) neighbors have the same sign, \( \pm 2 \) if all \( 3
    \) neighbors have the same sign, and \( 1 \) is opposite, and \( 0 \)
    if \( 2 \) of the neighbors have one sign, and the other \( 2 \) are
    opposite.  Then \( \Delta E \) can only have the values \( -16, -8,
    0, +8, +16 \).
\end{solution}

\begin{exercise}
    Compare the Metropolis and Glauber acceptance functions by drawing
    the graphs of the acceptance probabilities over the range of energy
    changes \( -17 \le \Delta E \le 17 \), one plot with the two
    functions for each of \( \text{k}_{\text{Boltz}} T = 10.0, 2.27,
    1.0, and 0.1 \).  Draw a dot on each graph at the possible values of
    \( \Delta T = -16, -8, 0, +8, +16 \)
\end{exercise}
\begin{solution}

\begin{asy}
import graph;

size(5inches, IgnoreAspect);

real myfontsize = 12;
real mylineskip = 1.2*myfontsize;
pen mypen = fontsize(myfontsize, mylineskip);
defaultpen(mypen);

real E(real DeltaT, real kBoltzT) { return exp(-DeltaT/kBoltzT);}
real Met(real DeltaT, real kBoltzT)
    { return min(1, E(DeltaT, kBoltzT)); }
real Glaub(real DeltaT, real kBoltzT)
    { return E(DeltaT, kBoltzT)/(1 + E(DeltaT, kBoltzT)); }

real[] DE = {-16, -8, 0, 8, 16};

picture p10 = new picture;
size(p10, 2inches, 2inches, IgnoreAspect);

real M10(real DeltaT) { return Met(DeltaT, 10); }
real G10(real DeltaT) { return Glaub(DeltaT, 10); }

draw(p10, graph(M10, -17, 17), red+1bp);
draw(p10, graph(G10, -17, 17), blue+1bp);
for (int i=0; i<5; ++i) {
  dot(p10, (DE[i], M10(DE[i])), red+3bp);
  dot(p10, (DE[i], G10(DE[i])), blue+3bp);
}

xaxis(p10, "$\Delta T$", xmin=-17, xmax=17, RightTicks); 
yaxis(p10, "Probability", ymin=0, ymax=1, XEquals(-17), Ticks); 

picture pTc = new picture;
size(pTc, 2inches, 2inches, IgnoreAspect);

real MTc(real DeltaT) { return Met(DeltaT, 2.27); }
real GTc(real DeltaT) { return Glaub(DeltaT, 2.27); }

draw(pTc, graph(MTc, -17, 17), red+1bp);
draw(pTc, graph(GTc, -17, 17), blue+1bp);
for (int i=0; i<5; ++i) {
  dot(pTc, (DE[i], MTc(DE[i])), red+3bp);
  dot(pTc, (DE[i], GTc(DE[i])), blue+3bp);
}

xaxis(pTc, "$\Delta T$", xmin=-17, xmax=17, RightTicks); 
yaxis(pTc, "Probability", ymin=0, ymax=1, XEquals(-17), Ticks); 

picture p1 = new picture;
size(p1, 2inches, 2inches, IgnoreAspect);

real M1(real DeltaT) { return Met(DeltaT, 1); }
real G1(real DeltaT) { return Glaub(DeltaT, 1); }

draw(p1, graph(M1, -17, 17), red+1bp);
draw(p1, graph(G1, -17, 17), blue+1bp);
for (int i=0; i<5; ++i) {
  dot(p1, (DE[i], M1(DE[i])), red+3bp);
  dot(p1, (DE[i], G1(DE[i])), blue+3bp);
}

xaxis(p1, "$\Delta T$", xmin=-17, xmax=17, RightTicks); 
yaxis(p1, "Probability", ymin=0, ymax=1, XEquals(-17), Ticks); 

picture p01 = new picture;
size(p01, 2inches, 2inches, IgnoreAspect);

real M01(real DeltaT) { return Met(DeltaT, 0.1); }
real G01(real DeltaT) { return Glaub(DeltaT, 0.1); }

draw(p01, graph(M01, -17, 17), red+1bp);
draw(p01, graph(G01, -17, 17), blue+1bp);
for (int i=0; i<5; ++i) {
  dot(p01, (DE[i], M01(DE[i])), red+3bp);
  dot(p01, (DE[i], G01(DE[i])), blue+3bp);
}

xaxis(p01, "$\Delta T$", xmin=-17, xmax=17, RightTicks); 
yaxis(p01, "Probability", ymin=0, ymax=1, XEquals(-17), Ticks); 

add(p10.fit(), (0,0), NW);
add(pTc.fit(), (0,0), NE);
add(p1.fit(), (0,0), SW);
add(p01.fit(), (0,0), SE);
\end{asy}
\end{solution}

\begin{exercise}
    Show that for both the Metropolis and Glauber Markov chains, all
    configurations are accessible, so the chain is irreducible.
\end{exercise}
\begin{solution}
    For the Metropolis chains with the deterministic visitation rule,
    consider configuration \( \omega \) which and configuration \(
    omega' \) which differs in one site.  It is possible to go from \(
    \omega \) to \( \omega' \) in at most \( N^2 \) steps, each of which
    has non-zero probability (although it may be small!) of occurring.
    That is, all configurations communicate, and the chain is
    irreducible.

    The same argument holds for the Glauber chain with the random
    selection visitation rule, although the expected time to reach one
    state from another is longer as in the coupon collectors problem,
    and the probability of movement is smaller.
\end{solution}

\hr

\visual{Books}{../../../../CommonInformation/Lessons/books.png}
\section*{Reading Suggestion:}

\bibliography{../../../../CommonInformation/bibliography}

%   \begin{enumerate}
%     \item
%     \item
%     \item
%   \end{enumerate}

\hr

\visual{Links}{../../../../CommonInformation/Lessons/chainlink.png}
\section*{Outside Readings and Links:}
\begin{enumerate}
    \item
    \item
    \item
    \item
\end{enumerate}

\section*{\solutionsname} \loadSolutions

\hr

\mydisclaim \myfooter

Last modified:  \flastmod

\end{document}

%%% Local Variables:
%%% mode: latex
%%% TeX-master: t
%%% TeX-master: t
%%% TeX-master: t
%%% End:
